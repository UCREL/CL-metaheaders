\chapter{Embedding}

\section{JSON}
JSON has no comment format, but is our native storage type, so a simple $``meta'':\{ ... \}$ inside the top-level object provides the entry point for the meta header.
In the (unlikely) event that the field ``meta'' is already used by the format in question, the alternative form $``\_\_meta\_\_''$ should be used.
Parsers working with `bare' JSON like this should attempt to read both types of block, and pick the one with a $\_\_version\_\_$ field if there is any ambiguity.

This is the only format that the meta header is \textit{not} inside a comment field, at present.

\section{WEKA ARFF}
Uses \% for comments, must be the first character on the line (may or may not ignore whitespace before \%).
The most direct way to include a meta header to this format is to simply encode the entire thing on one line to avoid parsing additional lines prefixed by \%`s.
However, to maintain readability, these will be supported in the finalised 1.0 specification.

To identify the meta header in among any other comments, the opening comment should be formatted thus; $\%!meta {...}$ and if the meta block spans multiple comment lines, only the first one should have the $!meta$ identifier.

\section{XML}
Uses the $<!-- ... -->$ format for comments, and can be placed anywhere, but some XML parsers might not support this and complain if the comment is the first thing in the file--incorrectly identifying the file as having multiple roots, despite not actually being a root element.
Rather than mandate that tools should update their parsers, we allow the following format anywhere in the first block of the file; $<!-- meta \{ ... \} -->$ wherein the standard then uses the normal JSON formatted data, and can span multiple lines.

