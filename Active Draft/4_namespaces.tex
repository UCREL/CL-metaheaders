\chapter{Namespaces}

Namespaces extend the basic meta block with additional functionality.
Developers are encouraged to create their own namespace for their own tool(s) if none of the currently accepted namespaces listed here apply.

To petition to have your namespace officially accepted, contact should be made through raising an issue at \url{https://github.com/UCREL/CL-metaheaders/issues} with a new feature request listing the following features of the proposed new namespace:

\begin{enumerate}
    \item The top-level name, as it should appear in the header
    \item Any field names that are included
    \item The current latest version, in Semantic Version\cite{semver} format
    \item The name contact details for the current maintainer
\end{enumerate}

Along with a clear description of what the namespace is used for, and any other details the submitter feels necessary.

If there is a specific tool or set of tools that uses the namespace described, details should be given of what the tool does, along with where the tool can be found, ideally as a web link to a stable address (GitHub pages are one such example, where the project URL is unlikely to change, but other sites are also acceptable).

\vspace{3em}\hrule\vspace{3em}

The rest of this section is given over to concrete implementations of namespaces recognised as 'standard' by this specification.
If you are planning to propose additional namespaces be added to the specification, the \texttt{history} namespace can be viewed as a template for the format and general content of your addition.

\clearpage\section{\texttt{history} - Operations History}

\begin{tabular}{r|l}
    Maintainer & John Vidler (\href{mailto:j.vidler@lancaster.ac.uk}{j.vidler@lancaster.ac.uk}) \\
    Version & 1.0.0 \\
    Last Updated & 12 June 2017 \\
\end{tabular} \vspace{1em}

A top-level namespace conforming to the \texttt{history} namespace specification that describes the processing history of this file. Complies with the interface specification and thus has an inner \texttt{\_\_version\_\_} field.

The complete structure for the \texttt{history} namespace is as follows (newlines inserted for clarity, and values skipped for brevity):

\begin{lstlisting}
{
    "__version__": "1.0.0",
    "log": [
        { "binary": "...", "time": "...", "args": "...", "platform": "...", "md5": "..." },
        { "binary": "...", "time": "...", "args": "...", "platform": "...", "md5": "..." }
    ]
}
\end{lstlisting}

Note that the $log$ field is an array of operations, which may contain any number of entries, including zero.


\subsection{\texttt{\_\_version\_\_}}
(String, Required)

The SemVer (see \ref{sec:SemanticVersioning}) compliant version that this history meta-block complies with, currently at 1.0.0.


\subsection{\texttt{log}}
(Array, Optional)

The array of logged actions, each entry should be a log-object.
If this field is absent, compliant parser implementations should interpret this as a zero-length array (no logged actions).
Each object in this array should have at least a $binary$ and $time$ field, but may also optionally include $args$, $platform$, and $md5$ fields as required.

\textbf{NOTE:} The log entries are likely to be an in-time-order array, but if you require guaranteed true time-order listing, ensure the parsing application sorts by the $time$ field.

\subsubsection{\texttt{binary}}
(String, Required)

The binary executed on this file.
If possible, non-system specific paths should be used to aid with compatibility between platforms.

\subsubsection{\texttt{time}}
(String, Required)

An ISO8601\cite{ISO/8601} format string for when the binary was executed.

\subsubsection{\texttt{args}}
(String, Optional)

Any command-line arguments that were passed to the binary for this log entry.
Quotes must be escaped in the string to comply with the JSON specification.

\subsubsection{\texttt{platform}}
(String, Optional)

The dot-delimited platform and architecture that this command was executed on.
Examples of possible values are shown below, but any valid $platform.arch$ string is permissible.

\begin{itemize}
    \item{Linux.x86}
    \item{Linux.x64}
    \item{Linux.ARMv7LE}
    \item{Linux.ARMv7BE}
    \item{Windows.x86}
    \item{Windows.x64}
    \item{osX.x86}
    \item{osX.x64}
\end{itemize}


\subsubsection{\texttt{md5}}
(String, Optional)

The MD5 hash string of the binary executed for this log entry.


\section{Example}

\lstinputlisting{examples/ns-history/example-json.json}

\noindent\textit{(NOTE: Linebreaks, indentation, and spacing added for readability but not required for the header to function)}